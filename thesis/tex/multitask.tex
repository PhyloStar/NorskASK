\chapter{Multitask learning}

So far we have considered \ac{AES} and \ac{NLI} as independent tasks and run
separate experiments. We will now try to train models to predict both tasks
simultaneously.

In table \ref{tab:multitask-results}, we see the results of running some of
the most successful models from the previous chapter in a multi-task setting,
using the essay's L1 as the auxiliary label to predict. The number following
the model name is the weight given to the auxiliary loss during training.

The reported metrics apply to the main task, CEFR prediction, only. It seems
that the auxiliary task is beneficial for all CNN models. All CNN models get
higher macro and micro \FI in the multi-task setup. Moreover, a higher weight
given to the auxiliary L1 prediction task seems to improve macro \FI
performance across the board. The effect of the auxiliary task weight seems
less clear on the micro \FI, but remember that in our setup, the micro \FI is
a side effect of the highest macro \FI seen during training.

\begin{table}
  \centering
  \begin{tabular}{lrrrr}
    \toprule
            & \multicolumn{2}{c}{All labels}       & \multicolumn{2}{c}{Collapsed labels} \\
    \cmidrule(lr){2-3}
    \cmidrule(lr){4-5}
    Model     & Macro \FI        & Micro \FI        & Macro \FI        & Micro \FI \\
    \midrule
    % $BEGIN autotable multitask-results
    % $META models-per-row=2 columns-per-model=macrof1,microf1
    % $ROW CNN1 0:      cnn-26094553_05          cnn-26094553_06
    % $ROW CNN1 0.25:   cnn-multi-26199963_1     cnn-multi-26199963_2
    % $ROW CNN1 0.5 :   cnn-multi-26199963_3     cnn-multi-26199963_4
    % $ROW CNN1 0.75:   cnn-multi-26199963_5     cnn-multi-26199963_6
    % \midrule
    % $ROW CNN2 0:      cnn-26094553_11          cnn-26094553_12
    % $ROW CNN2 0.25:   cnn-multi-26199963_7     cnn-multi-26199963_8
    % $ROW CNN2 0.5 :   cnn-multi-26199963_9     cnn-multi-26199963_10
    % $ROW CNN2 0.75:   cnn-multi-26199963_11    cnn-multi-26199963_12
    % \midrule
    % $ROW RNN1 0:      rnn-25858209             rnn-25858211
    % $ROW RNN1 0.25:   rnn-multi-26199963_13    rnn-multi-26199963_14
    % $ROW RNN1 0.5 :   rnn-multi-26199963_15    rnn-multi-26199963_16
    % $ROW RNN1 0.75:   rnn-multi-26199963_17    rnn-multi-26199963_18
    % \midrule
    % $ROW RNN2 0:      rnn-25985451             rnn-25985452
    % $ROW RNN2 0.25:   rnn-multi-26199963_19    rnn-multi-26199963_20
    % $ROW RNN2 0.5 :   rnn-multi-26199963_21    rnn-multi-26199963_22
    % $ROW RNN2 0.75:   rnn-multi-26199963_23    rnn-multi-26199963_24
    % $END autotable
    CNN1 0 & $0.235$ & $0.382$ & $0.404$ & $0.699$ \\
    CNN1 0.25 & $0.233$ & $0.439$ & $0.380$ & $0.715$ \\
    CNN1 0.5 & $0.246$ & $0.398$ & $0.389$ & $0.732$ \\
    CNN1 0.75 & $0.247$ & $0.431$ & $0.408$ & $0.707$ \\
    \midrule
    CNN2 0 & $0.255$ & $0.447$ & $0.393$ & $0.740$ \\
    CNN2 0.25 & $0.259$ & $0.455$ & $0.406$ & $0.764$ \\
    CNN2 0.5 & $0.270$ & $0.415$ & $0.398$ & $0.748$ \\
    CNN2 0.75 & $0.275$ & $0.463$ & $0.414$ & $0.780$ \\
    \midrule
    RNN1 0 & $\mathbf{0.354}$ & $0.390$ & $0.493$ & $0.724$ \\
    RNN1 0.25 & $0.301$ & $0.431$ & $0.506$ & $0.797$ \\
    RNN1 0.5 & $0.333$ & $0.472$ & $0.529$ & $0.772$ \\
    RNN1 0.75 & $0.283$ & $\mathbf{0.488}$ & $0.526$ & $0.772$ \\
    \midrule
    RNN2 0 & $0.277$ & $0.407$ & $\mathbf{0.624}$ & $0.756$ \\
    RNN2 0.25 & $0.320$ & $0.423$ & $0.565$ & $0.683$ \\
    RNN2 0.5 & $0.292$ & $0.447$ & $0.533$ & $0.772$ \\
    RNN2 0.75 & $0.285$ & $0.447$ & $0.428$ & $\mathbf{0.805}$ \\
    \bottomrule
  \end{tabular}
  \caption{CNN1: Static, pretrained embeddings size 50.
           CNN2: Static, pretrained embeddings size 50, POS as side input.
           RNN1: Attention model with GRU cell, dynamic pretrained
           embeddings size 50.
           RNN2: Attention model with LSTM cell, dynamic pretrained
           embeddings size 50, POS as side input.}
  \label{tab:multitask-results}
\end{table}

\begin{table}
  \centering
  \begin{tabular}{lll}
    \toprule
    Hyperparameter & CNN1 & CNN2 \\
    \midrule
    Word embeddings & Static & Static \\
    Embedding init & Pre-trained & Pre-trained \\
    Embedding size & 50 & 50 \\
    Side input & None & UPOS \\
    \bottomrule
  \end{tabular}
  \caption{Descriptions of the two CNN models}
  \label{tab:cnn-parameters}
\end{table}

\begin{table}
  \centering
  \begin{tabular}{lll}
    \toprule
    Hyperparameter & RNN1 & RNN2 \\
    \midrule
    Word embeddings & Dynamic & Dynamic \\
    Embedding init & Pre-trained & Pre-trained \\
    Embedding size & 50 & 50 \\
    RNN cell & GRU & LSTM \\
    Side input & None & UPOS \\
    \bottomrule
  \end{tabular}
  \caption{Descriptions of the two RNN models}
  \label{tab:rnn-parameters}
\end{table}

For the RNNs, the results are not quite as clear. The multi-task models were
not able to exceed the macro \FI scores of $0.354$ (all labels) and $0.624$
(collapsed labels), but this may be because the macro \FI metric is unstable
because of the `A2' class with one single example in the dev set. For the
same models, accuracy did increase in the multitask setup.

The model `RNN1 0.5' was the best performing multitask model by several
metrics. In addition to macro \FI, it was the best in terms of RMSE
($0.888$), MAE ($0.610$), Pearson's correlation coefficient ($0.765$) and
Spearman's ranked correlation coefficient ($0.768$).


\begin{figure}
  % cnn-multi-26199963_11
  \centering
  \includegraphics[width=\textwidth]{cnn-multi-training}
  \caption{CNN2 0.75. Training and validation loss and accuracy over 50 epochs of training.
           Also, confusion matrices with counts and normalized by class.}
  \label{fig:cnn-multi-training}
\end{figure}


\begin{figure}
  % rnn-multi-26199963_15
  \centering
  \includegraphics[width=\textwidth]{rnn-multi-training}
  \caption{RNN1 0.5. Training and validation loss and accuracy over 50 epochs of training.
           Also, confusion matrices with counts and normalized by class.}
  \label{fig:rnn-multi-training}
\end{figure}
