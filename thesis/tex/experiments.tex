
\section{Preprocessing}

The data files in the ASK corpus are in XML format, and contain information
about tags, mistakes and corrections, paragraphs, sentences and more. These
files were transformed into several other formats during the process. First,
they were converted to plain text files, stripped of all tags or correction
labels. The text files have one sentence per line, consisting of
space-separated tokens, and an empty line separating paragraphs.

These raw text files were then sent through the UDPipe pipeline
\autocite{udpipe:2017} for tagging and dependency parsing. The UDPipe
project maintains a REST API containing a selection of pretrained models.
All documents were transformed by the REST API using what was at the time 
of writing the
newest Norwegian bokmål (nb) model, available, namely
\texttt{norwegian-bokmaal-ud-2.3-181115}.

The pipeline accepts raw text files as input, where each sentence is put on a
separate line. The output from UDPipe is in the CoNLL file format, with a
single token per line. UDPipe tags the documents using the UD tagset, while
the original tags in the XML documents are from the Oslo-Bergen tagger's own
tagset.

\section{Metrics}

Because of the unbalanced nature of the classes in the dataset, special
consideration has to be made as to the metrics of evaluation. Two metrics are
reported for all experiments: The macro average F1 and the micro average F1.
The latter is equivalent to the accuracy (ratio of correctly predicted
samples), while the former gives all the classes equal weight. If a model
correctly classifies most samples in the classes with most samples, but gives
poor predictions for smaller classes, then the macro F1 score for the model
will be considerably lower than the micro F1 score.

The metrics are reported for two different modes: The first utilizing the
full set of classes, and the second training and evaluating on the collapsed
classes.

A third option, namely to train on the full set of classes and reduce the
predictions to the collapsed set of classes, was also attempted, based on the
assumption that the more fine grained labels in the full set of classes can
provide useful supervision signals even though we evaluate on a smaller set.
However, in practice the best perforers on the collapsed labels was
empirically observed to be the models that were also trained on the collapsed
tags.

\section{Model descriptions}

Two models.

\subsection{Logistic regression}

The logistic regression model was implemented using the Python
library Scikit-Learn.

\section{Results}

Two different sets of classes are used in the experiments. The original seven
CEFR labels, and a collapsed set where the intermediate classes, such as
``A2/B1'', are rounded up to the nearest canonical class, i.e. the CEFR label
right of the slash. This results in only four different labels: ``A2'',
``B1'', ``B2'' and ``C1''.

The majority class in the training set is ``B1'', whether we consider the
full class set or the collapsed set. A majority classifier gets an
accuracy of 18.7\% on the test set using non-collapsed labels. With the
collapsed labels, the accuracy on the test set is 34.1\%. The macro F1
scores are much lower, as the majority class classifier predicts no
samples for any other classes.

Moving to a linear model, a logistic regression classifier using only
bag-of-word features achieves an accuracy of 28.5\% on the dev set without
collapsed labels and 58.5\% with collapsed labels. There are approximately
18,300 different word forms in the training set, and therefore the same
number of features in the bag-of-word model.

Several neural network models were attempted as well, with input either being
word counts, character \ngrams, or part of speech \ngrams. For the \ngram
features, \textit{n}s in the interval [2, 4] were used. For all models, only
the 10,000 most common features were kept, unlike the linear classifier which
used the full vocabulary in the training set.

All results are seen in table \ref{baseline-accuracies}.

\begin{table}
  \centering
  \begin{tabular}{|l|rr|rr|}
    \toprule
      & \multicolumn{2}{c|}{All labels} & \multicolumn{2}{c|}{Collapsed labels} \\
    Model      & Macro F1 & Micro F1 & Macro F1 & Micro F1 \\
    \midrule
    Majority   &    4.0\% &   16.3\% &   12.7\% &   34.1\% \\
    LogReg BOW &   17.9\% &   28.5\% &   31.9\% &   58.5\% \\
    MLP BOW    &   22.1\% &   35.0\% &   41.0\% &   64.2\% \\
    MLP Char   &   20.3\% &   36.6\% &   36.8\% &   70.7\% \\
    MLP POS    &   24.5\% &   39.0\% &   42.8\% &   65.9\% \\
    \bottomrule
  \end{tabular}
  \caption{F1 scores of different classifiers}
  \label{baseline-accuracies}
\end{table}

The part of speech \ngrams performed best overall, having the highest F1 score
for the full label set, both for macro and micro average. On the collapsed set
of labels, the part of speech \ngrams had the highest macro F1 score, but were
beat in micro F1 by the character \ngrams.
