
\section{Preprocessing}

The data files in the ASK corpus are in XML format, and contain information
about tags, mistakes and corrections, paragraphs, sentences and more. These
files were transformed into other formats during the process. First, they
were converted to plain text files stripped of all tags or correction labels,
with one sentence per line consisting of space-separated tokens, and an empty
line separating paragraphs.

These raw text files were then sent through the UDPipe pipeline for tagging
and dependency parsing. The output from UDPipe is in the CoNLL file format
with a single token per line. UDPipe tags the documents using the UD tagset,
while the original tags in the XML documents are from the Oslo-Bergen
tagger's own tagset.

\section{Baseline}

The majority class in the training set is B1. A majority classifier gets an
accuracy of 18.7\% on the test set.

Moving to a simple bag-of-words model, a logistic regression classifier achieves
an accuracy of ??? on the dev set. 

A logistic regression classifier was able to predict the CEFR score with
43.9\% accuracy using only two features: The length of the document, in
number of tokens, and the test level (IL test or AL test). 54 of 123
documents in the dev set.

A convolutional neural network based on the architecture in
\textcite{zhang2017sensitivity} achieved an accuracy of 42.3\% using
sequences of POS tags as input to an initial embedding layer.

Note that always predicting the majority class in the dev set (B2 with 37
documents) yields an accuracy of 30.1\%.
