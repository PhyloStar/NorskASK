
\section{Preprocessing}

The data files in the ASK corpus are in XML format, and contain information
about tags, mistakes and corrections, paragraphs, sentences and more. These
files were transformed into other formats during the process. First, they were
converted to plain text files stripped of all tags or correction labels, with
one sentence per line consisting of space-separated tokens, and an empty line
separating paragraphs.

These raw text files were then sent through the UDPipe pipeline for tagging and
dependency parsing. The output from UDPipe is in the CONLL file format with
a single token per line. UDPipe tags the documents using the UD tagset, while
the original tags in the XML documents are from the Oslo-Bergen tagger's own
tagset.
