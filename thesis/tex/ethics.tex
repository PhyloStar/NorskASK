\chapter{Ethical considerations}

Language testing is a high-stakes setting. Often, an official language
proficiency certificate is needed in order to be admitted to a study
programme or gain employment, for instance. The high stakes involved means
that the grading of a language test should ideally be transparent and
explainable, something many modern machine learning based models struggle to
achieve. In the worst case, a person may fail an official language test that
is graded automatically, and have no way to know what aspects of the test
caused them to fail.

Even if a computer essay grading system is not used for grading an official
test, it may indirectly influence a language learner's decision to take a
test at a certain time. As established, language testing can be inconvenient
for those taking it, since they have to get to the testing location, pay a
fee, etc. For instance, if a language learner uses an automatic grading
system to find out what CEFR level they are at, and it concludes that they
are at a B2 level, while in fact the learner is still at a lower level, they
may choose to take a B2 level language test they are likely to fail. On the
other hand, if the grading system undershoots, the learner may waste time by
not taking the test at a point in time where they would already be ready for
it.

Language testing is in some places a requirement for gaining citizenship, a
practice that has been criticized [citation needed].

Native language identification also raises ethical concerns. Assumptions
about the dependence of native language and country of origin has been used
as arguments in asylum cases. These assumptions are often wrong, and if
automatic native language identification technologies are easily accessible
they may increase the occurrence of this practice. This needs better
references, but here's
\href{https://en.wikipedia.org/wiki/Language_analysis_for_the_determination_of_origin}{Wikipedia
on \ac{LADO}}.
