\section{Ethical considerations}

Language testing is a high-stakes setting. Often, an official language
proficiency certificate is needed in order to be admitted to a study
programme or gain employment, for instance. The high stakes involved means
that the grading of a language test should ideally be transparent and
explainable, something many modern machine learning based models struggle to
achieve.

Language testing is in some places required to gain citizenship, a practice
that has been criticized [citation needed].

Native language identification also raises ethical concerns. Assumptions
about the dependence of native language and country of origin has been used
as arguments in asylum cases. These assumptions are often wrong, and if
automatic native language identification technologies are accessible they may
increase the occurrence of this practice.
